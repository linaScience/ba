\begin{document}


- restriction on time series models?
    - ARMA, ARIMA, ...
    - Multilayer Perceptrons
    - research is given based on prediction on RSSI 
    - location of aps is unknown
- 1st place of competition used kNN (distance error of 5.8) and LightGBM (distance error of 7.25)
- simplification: Mobile Terminal is traveling with constant speed (although acceleration is given)

- Karl ist freund vom si package

Intro:

    Content:
        - SSF and LLF is not sufficient for predicting movement
        - current work
        - Main question: "How could we improve handover for Access Points with predicting movement of a mobile terminal?"
        - Thesis covers:
            - explanation of the choice of the data set
            - prediction of movement of an Mobile Terminal "indoors"
                - using wifi and waypoint data, given in specific dataset -> filtering
                - separation of train data (? -> sinnvoll, test data hat zwar auch keine waypoints, aber validierung könnte hier schwieriger sein?)
            - assessment and restriction of general ML algorithm type
            - implementation of movement prediction with ML algorithm
            - ...
        - thesis does not cover:
            - creation of an own and most suitable dataset and usage in ml prediction
            - implementation of handover prediction mechanism
            - using more data like magnetic field etc.
            - general new conception of a ML algorithm for general forecasting purposes
            - prediction in

Grundlagen/definitorischer Teil:
    - problemorientierte Definitionen definieren
    - Überblick in der Literatur vorhandenen Methoden bzw. Lösungsansätze
    - aktueller Stand der Technik
    - Related Work

- time based/sequential data
- ...

Main part:
    - eigentliche Auseinandersetzung mit der Problemstellung
        - Problemstellung reflektiert bearbeiten
        - Aussagen durch Literatur stützen und belegen
        - logisch und nachvollziehbare Schritte

    Ideas for ML Algorithms:
        - time based model
            - only when data is (mostly even) distributed over time
            - ARIMA, ARMA, ...
        - models with sequential data
            - LSTM
            - RNN
            - GRU
        - others
            - kNN

Approach:
    - filtering of data in dataset for TYPE_WIFI and TYPE_WAYPOINT, for prediction
        - analysis on data
        - find out how we can forecast the movement of the user with this data
        - we do not need to filter for specific SSIDs, because we only want to predict the waypoint and not a possible roam (could be part of future work)
    - analysis of current ML algorithms and choosing "the right one"
        - we have 27,000 train data and about 600 test data but splitting the actual train data into test data could be important
        - first we need to approximate the location of the wifi APs, which we can do with kNN (necessary?)
            - kNN is a supervised learning algorithm
            - find a proper distance function
            - we can use the rssi for kNN
        - next we want to find the next waypoint based on the wifi data 
            - we want supervised learning, because we have the waypoints given in train data but not in test data
                - we want to forecast the next waypoint based on the wifi data
                    - we want to use past values to predict the next waypoint (univariate time series)
                    - Naïve models: XGBoost, ARIMA; Deep Learning models: RNN, LSTM, MLP
                    - ARIMA: autocorrelation relates on same set of observations but across different time; we have many different observations, that's why ARIMA is not suitable
                    - RNN: backpropagation could be a problem (gradient vanishing and exploding problems could occur), RNN suffers from short-term memory (cannot process long sequences, uses tanh as activation function)
                    - LSTM/GRU: uses previous data with context 
        - error function for penalty of wrong prediction
            - could be quadratic error function, because the more the station is away from the ap, the worse 
        - cross-validation (k-Fold most common)

    - implementation of Machine Learning Algorithm
        - what do i have?
            - data with time and specific type like TYPE_WIFI and TYPE_WAYPOINT in train dataset
            - separate data for validation: 80/20 or 70/30 for testing and remove the waypoints
        - what do i need?
            - ml model(s) which is suitable for the data
                - kNN for predicting the position of the station based on the rssi
                    - proper distance function between two WiFi observations, and compare WiFi observations at inference time with all WiFi observations seen during training
                -  use gru for predicting the next waypoint based on the rssi
                - working hypothesis:
                    - prediction without the rssi is worse than with
            - resources of my PC


Schlussbetrachtung:
    - Antwort auf in Problemstellung aufgeworfenen Fragen kurz zusammenfassen
    - Ausblick auf offen gebliebene Fragen sowie interessante Fragestellungen, die sich aus der Arbeit ergeben
    - kritische Betrachtung eigener Arbeit

\end{document}