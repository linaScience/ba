\chapter{Related Work}\label{ch:related-work}

Montavont et al. \cite{handover-assisted-by-gps} propose a handover decision algorithm based on the \ac{gps} location of the mobile device.
Khan et al. \cite{MLBasedHandoverPrediction2022} address the problem of handover prediction and \ac{ap} selection in dense \ac{wifi} networks with \ac{sdn}.
They use \ac{ml} models such as \ac{svr} and \ac{mlp} for estimating the throughput of the network to.
The predictions outperform the current approaches of strongest received signal first by 9.2\% and least loaded signal first by 8\%.

Both approaches do not consider the trajectory of the mobile device in the handover or \ac{ap} selection.
Furthermore, in large-scale and dense \ac{wifi} environments, \ac{gps} may not be available or not accurate enough for indoor trajectories.
While Khan et al. focuses on using \ac{ml} for throughput estimation of the network and accordingly choose the best \ac{ap} to roam to, they do not consider the trajectory of the mobile device in the \ac{ap} selection or use \ac{ml} for the \ac{ap} selection process directly.
Lastly, these approaches do not use user generated data, but synthesized data.

This thesis however will use user generated data, instead of synthesized, and will use \ac{ml} to predict the next \ac{ap} a user will be nearest to based on the \ac{rssi} of the \acp{ap} and the trajectory of the user.
The prediction will be difficult, because the dataset does not provide the location of the \acp{ap}, and factors such as walls and other obstacles may influence the \ac{rssi}.
Additionally, there is no information about the throughput or load an access point has, so the predictions cannot be use this information.

These predictions can be used to improve the handover process in \ac{wifi} networks initiated by a mobile device, because the predictions use real-world user generated data.
This thesis has a different approach, so the results of this thesis cannot be compared to the related work.
