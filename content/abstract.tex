\null\vfil
\begin{otherlanguage}{english}
\begin{center}\textsf{\textbf{\abstractname}}\end{center}

\noindent 
Mobile device roaming on \ac{wifi} networks currently does not consider human movement, leading to performance issues.
This thesis presents a machine learning approach that uses an LSTM model to predict the nearest \ac{ap} based on \ac{rssi} values from the surrounding \ac{ap}.
The model was trained using real-world data from a Microsoft Research competition and achieved a top 3 prediction accuracy of 76\%.
However, the performance was only marginally better than a heuristic method.
Improvements could be achieved with fewer classes, a different sized sliding window, additional sensor data.
Future research should focus on generated data with more information such as the location of the \ac{ap}, so that human movement can be utilized to precisely predict the next \ac{ap}.
\end{otherlanguage}
\vfil\null


% => Wenn die Arbeit auf Englisch verfasst wurde, verlangt das Studienreferat einen englischen UND deutschen Abstract (der dt. Abstract kann dann ggf. auch ans Ende der Arbeit)

% deutsche Zusammenfassung
\null\vfil
\begin{otherlanguage}{ngerman}
\begin{center}\textsf{\textbf{\abstractname}}\end{center}

\noindent 
Beim Roaming mobiler Geräte in \ac{wifi}-Netzwerken werden menschliche Bewegungen derzeit nicht berücksichtigt, was zu Leistungsproblemen führt.
Diese Arbeit stellt einen maschinellen Lernansatz vor, der ein \ac{lstm}-Modell verwendet, um den nächsten \ac{ap} basierend auf \ac{rssi}-Werten aus dem umgebenden \ac{ap} vorherzusagen.
Das Modell wurde anhand realer Daten aus einem Microsoft Research-Wettbewerb trainiert und erreichte eine Top-3-Vorhersagegenauigkeit von 76\%.
Allerdings war die Leistung nur unwesentlich besser als bei einer heuristischen Methode.
Verbesserungen könnten durch weniger Klassen, ein unterschiedlich großes Schiebefenster und zusätzliche Sensordaten erreicht werden.
Zukünftige Forschungen sollten sich auf generierten Daten mit mehr Informationen wie der Position des \ac{ap} konzentrieren, damit menschliche Bewegungen zur präziseren Vorhersage des nächsten \ac{ap} genutzt werden können.
\end{otherlanguage}
\vfil\null