\null\vfil
\begin{otherlanguage}{english}
\begin{center}\textsf{\textbf{\abstractname}}\end{center}

\noindent 
Mobile device roaming on \ac{wifi} networks currently does not consider human movement, leading to performance issues for e.g. video conferencing \cite{handoff_performance_issues}.
This thesis presents a machine learning approach that uses an \ac{lstm} model to predict the nearest \ac{ap} based on \ac{rssi} values from the surrounding \acp{ap}\footnote{Implementation of this thesis: \url{https://github.com/linaScience/ba-implementation}}.
The models input are time sequences of waypoint, acceleration and \ac{wifi} data such as \acp{bssid} with their \ac{rssi} values.
The output of the model will be used to get the top three \acp{bssid}, where one of it must be the \ac{ap} the station connects to next to be a correct prediction.
The model was trained using modified real-world data from a Microsoft Research competition and achieved a top three prediction accuracy of 76\%.
With these predictions, the roaming process in \ac{wifi} networks initiated by a mobile device could be improved.
However, the performance was only marginally better than a heuristic method.
Further improvements could be achieved with fewer classes, a different sized sliding window, and additional sensor data.
Future research should focus on generated data with more information such as the location of the \ac{ap}, so that human movement can be utilized to precisely predict the next \ac{ap}.
\end{otherlanguage}
\vfil\null


% => Wenn die Arbeit auf Englisch verfasst wurde, verlangt das Studienreferat einen englischen UND deutschen Abstract (der dt. Abstract kann dann ggf. auch ans Ende der Arbeit)

% deutsche Zusammenfassung
\null\vfil
\begin{otherlanguage}{ngerman}
\begin{center}\textsf{\textbf{\abstractname}}\end{center}

\noindent
Das Roaming von Mobilgeräten in \ac{wifi}-Netzwerken berücksichtigt derzeit nicht die Bewegungen von Menschen, was zu Leistungsproblemen, z. B. bei Videokonferenzen, führt \cite{handoff_performance_issues}.
In dieser Arbeit wird ein Ansatz des maschinellen Lernens vorgestellt, der ein \ac{lstm}-Modell zur Vorhersage des nächstgelegenen \ac{ap} auf der Grundlage von \ac{rssi}-Werten aus den umliegenden \acp{ap}.
Die Eingabe des Modells sind Zeitsequenzen von Wegpunkt-, Beschleunigungs- und \ac{wifi}-Daten wie \acp{bssid} mit ihren \ac{rssi}-Werten.
Die Ausgabe des Modells wird verwendet, um die drei besten \acp{bssid} zu erhalten, von denen einer der \ac{ap} sein muss, mit dem sich die Station als nächstes verbindet, um eine korrekte Vorhersage zu erhalten.
Das Modell wurde mit modifizierten realen Daten aus einem Microsoft Research-Wettbewerb trainiert und erreichte eine Vorhersagegenauigkeit für die ersten drei Plätze von 76\%.
Mit diesen Vorhersagen könnte der Roaming-Prozess in \ac{wifi}-Netzen, der von einem mobilen Gerät initiiert wird, verbessert werden.
Die Leistung war jedoch nur geringfügig besser als eine heuristische Methode.
Weitere Verbesserungen könnten mit weniger Klassen, einer anderen Größe des Schiebefensters und zusätzlichen Sensordaten erzielt werden.
Zukünftige Forschungen sollten sich auf die Generierung von Daten mit mehr Informationen, wie z. B. dem Standort des \ac{ap}, konzentrieren, sodass die menschliche Bewegung zur genauen Vorhersage des nächsten \ac{ap} genutzt werden kann.
\end{otherlanguage}
\vfil\null