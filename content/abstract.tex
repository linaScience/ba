\null\vfil
\begin{otherlanguage}{english}
\begin{center}\textsf{\textbf{\abstractname}}\end{center}

\noindent 
Mobile device roaming on \ac{wifi} networks currently does not consider human movement, leading to performance issues for e.g. video conferencing \cite{handoff_performance_issues}.
This thesis presents a machine learning approach that uses an LSTM model to predict the nearest \ac{ap} based on \ac{rssi} values from the surrounding \ac{ap}.
The models input are sequences of waypoint, acceleration and \ac{wifi} data such as \acp{bssid} with their \ac{rssi} values.
Its output is the top 3 \acp{bssid} the model predicts the station will connect to next.
The model was trained using modified real-world data from a Microsoft Research competition and achieved a top 3 prediction accuracy of 76\%.
With these predictions, the roaming process in \ac{wifi} networks initiated by a mobile device could be improved.
However, the performance was only marginally better than a heuristic method.
Improvements could be achieved with fewer classes, a different sized sliding window, additional sensor data.
Future research should focus on generated data with more information such as the location of the \ac{ap}, so that human movement can be utilized to precisely predict the next \ac{ap}.
\end{otherlanguage}
\vfil\null


% => Wenn die Arbeit auf Englisch verfasst wurde, verlangt das Studienreferat einen englischen UND deutschen Abstract (der dt. Abstract kann dann ggf. auch ans Ende der Arbeit)

% deutsche Zusammenfassung
\null\vfil
\begin{otherlanguage}{ngerman}
\begin{center}\textsf{\textbf{\abstractname}}\end{center}

\noindent 
Beim Roaming mobiler Geräte in \ac{wifi}-Netzwerken werden menschliche Bewegungen derzeit nicht berücksichtigt, was zu Leistungsproblemen z. B. bei mobilen Endgeräten führt. Videokonferenzen \cite{handoff_performance_issues}.
Diese Arbeit stellt einen maschinellen Lernansatz vor, der ein LSTM-Modell verwendet, um den nächsten \ac{ap} basierend auf \ac{rssi}-Werten aus dem umgebenden \ac{ap} vorherzusagen.
Die Modelleingaben sind Sequenzen von Wegpunkt-, Beschleunigungs- und \ac{wifi}-Daten wie \acp{bssid} mit ihren \ac{rssi}-Werten.
Seine Ausgabe sind die Top 3 \acp{bssid}, mit denen das Modell vorhersagt, dass die Station als nächstes eine Verbindung herstellen wird.
Das Modell wurde mithilfe modifizierter realer Daten aus einem Microsoft Research-Wettbewerb trainiert und erreichte eine Top-3-Vorhersagegenauigkeit von 76 %.
Mit diesen Vorhersagen könnte der von einem Mobilgerät initiierte Roaming-Prozess in \ac{wifi}-Netzwerken verbessert werden.
Allerdings war die Leistung nur unwesentlich besser als bei einer heuristischen Methode.
Verbesserungen könnten durch weniger Klassen, ein unterschiedlich großes Schiebefenster und zusätzliche Sensordaten erreicht werden.
Zukünftige Forschung sollte sich auf generierte Daten mit mehr Informationen wie der Position des \ac{ap} konzentrieren, damit menschliche Bewegungen genutzt werden können, um den nächsten \ac{ap} präzise vorherzusagen.
\end{otherlanguage}
\vfil\null