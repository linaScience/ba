\null\vfil
\begin{otherlanguage}{english}
\begin{center}\textsf{\textbf{\abstractname}}\end{center}

\noindent 
Mobile device roaming on \ac{wifi} networks currently does not consider human movement, leading to performance issues, e.g., video conferencing \cite{handoff_performance_issues}.
This thesis presents a \ac{ml} approach that uses an \ac{lstm} model to predict the nearest \ac{ap} based on \ac{rssi} values from the surrounding \acp{ap}\footnote{Implementation of this thesis: \url{https://github.com/linaScience/ba-implementation}}.
The model's inputs are time sequences of waypoint, acceleration, and \ac{wifi} data such as \acp{bssid} with their \ac{rssi} values.
The output of the model will be used to get three \acp{bssid}, named \threeAP, where one of them must be the following \ac{ap} the station may connect to be a correct prediction.
The model was trained using modified real-world data from a Microsoft Research competition and achieved a prediction accuracy of 76\,\%.
With these predictions, the roaming process in \ac{wifi} networks initiated by a mobile device could be improved.
However, the performance was only marginally better than a heuristic method, which selects the \acp{bssid} with the highest \ac{rssi} from the latest measured \ac{rssi} values.
Further improvements could be achieved with fewer classes, a different-sized sliding window, and additional sensor data.
Future research should focus on generated data with more information, such as the location of the \ac{ap} so that human movement can precisely predict the following \ac{ap}.
\end{otherlanguage}
\vfil\null


% => Wenn die Arbeit auf Englisch verfasst wurde, verlangt das Studienreferat einen englischen UND deutschen Abstract (der dt. Abstract kann dann ggf. auch ans Ende der Arbeit)

% deutsche Zusammenfassung
\null\vfil
\begin{otherlanguage}{ngerman}
\begin{center}\textsf{\textbf{\abstractname}}\end{center}

\noindent
Das Roaming von Mobilgeräten in \ac{wifi}-Netzwerken berücksichtigt derzeit nicht die Bewegungen von Menschen, was zu Leistungsproblemen, z. B. bei Videokonferenzen, führt.
In dieser Arbeit wird ein \ac{ml}-Ansatz vorgestellt, der ein \ac{lstm}-Modell zur Vorhersage des nächstgelegenen \ac{ap} auf der Grundlage von \ac{rssi}-Werten aus dem umgebenden \acp{ap}.
Die Eingaben des Modells sind Zeitsequenzen von Wegpunkt-, Beschleunigungs- und \ac{wifi}-Daten wie \acp{bssid} mit ihren \ac{rssi}-Werten.
Die Ausgabe des Modells wird verwendet, um drei \acp{bssid}, genannt \threeAP, zu erhalten, von denen eine die folgende \ac{ap} sein muss, die die Station verbinden kann, um eine korrekte Vorhersage zu sein.
Das Modell wurde mit modifizierten realen Daten aus einem Microsoft Research-Wettbewerb trainiert und erreichte eine Vorhersagegenauigkeit von 76\,\%.
Mit diesen Vorhersagen konnte der Roaming-Prozess in \ac{wifi}-Netzen, die von einem mobilen Gerät initiiert werden, verbessert werden.
Die Leistung war jedoch nur geringfügig besser als eine heuristische Methode, die aus den zuletzt gemessenen \ac{rssi}-Werten die \acp{bssid} mit dem höchsten \ac{rssi} auswählt.
Weitere Verbesserungen könnten mit weniger Klassen, einem anders großen Schiebefenster und zusätzlichen Sensordaten erreicht werden.
Zukünftige Forschungen sollten sich auf die Generierung von Daten mit mehr Informationen konzentrieren, wie z. B. den Standort des \ac{ap}, damit die menschliche Bewegung das folgende \ac{ap} genau vorhersagen kann.
\end{otherlanguage}
\vfil\null

\acresetall
