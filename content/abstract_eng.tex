\null\vfil
\begin{otherlanguage}{english}
\begin{center}\textsf{\textbf{\abstractname}}\end{center}

\noindent 
\begin{itemize}
    \item human movement not considered when roaming of mobile devices appears
    \item could lead to many unnecessary handovers and thus to bad performance
    \item this thesis proposes a machine learning-based approach to predict the \ac{ap} a user is nearest to based on the \ac{rssi} of the \acp{ap}
    \item the chosen machine learning model is \ac{lstm} and is trained on real-world data from a dataset of a competition by Microsoft Research
    \item the dataset analysis showed that interpolation is necessary to ensure that \ac{wifi} and waypoint data together are used in the prediction
    \item many \acp{ap} are deployed in the buildings of the dataset, which makes the prediction task hard
    \item one building and one floor were chosen to be the dataset for this thesis
    \item the evaluation of the model shows that the mode predicts the \ac{ap} a user is nearest to with a top 3 accuracy of about 71\%
    \item too many classes, which makes prediction too hard and top 3 accuracies too low
    \item with fewer classes, the model could predict better
    \item in the future, data explicitly generated for this purpose could be used to predict the top k access points so that the location of the access points is known and can be considered in prediction
\end{itemize}

\end{otherlanguage}
\vfil\null
