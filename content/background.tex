\chapter{Background}\label{sec:background}

\section{Machine Learning in Wireless Networks}
As Szott et al. noted in their survey\cite{szottWiFiMeetsML2022}, ML is also now utilized in \acp{wlan}.
For \ac{wifi}, a prediction for a possible roam could initiate the roaming process sooner, thereby improving the user experience and overall \ac{wifi} network performance.

% What is time-series prediction
\section{Time Series Prediction}
Time series data, comprising a sequence of data points ordered in time, represents a common structure in many domains, including user mobility within a Wi-Fi network.
Owing to its inherent temporal dependencies—where subsequent data points can be influenced by previous ones—particular machine learning techniques are typically employed.
These include the \ac{arima} model, and \ac{rnn} models such as \ac{lstm} and \ac{gru}.
Each of these techniques is designed to capture and leverage temporal patterns within the data, facilitating the prediction of future trends based on historical observations.

In the context of machine learning model development, the configuration of hyperparameters represents a crucial task.
Defined as the set of parameters that govern the learning process and are not learned from the data, hyperparameters encompass elements such as the learning rate, the number of layers within a neural network, and the number of clusters in a clustering algorithm.
As these parameters are determined a priori, their careful selection—known as hyperparameter tuning or optimization—is necessary to maximize model performance.
This iterative procedure involves exploring various hyperparameter combinations in search of the configuration that yields the most accurate predictions.

Moreover, the evaluation of data stands as an essential component of any machine learning project, and this remains true for time series prediction.
Such evaluation involves an assessment of the model's performance by contrasting predicted outcomes with actual results.
One common technique is cross-validation, wherein the data set is partitioned into a training set for model training, and a validation set for model evaluation.
Performance evaluation is indispensable for understanding the model's accuracy, reliability, and ability to generalize to new, unseen data.
Furthermore, it provides insights into potential underfitting, where the model fails to learn sufficient patterns from the training data, or overfitting, where the model becomes overly sensitive to noise or outliers in the training data, both of which can significantly impair predictive performance.

% What is supervised learning
\section{Supervised Learning}
Supervised learning is a \acs{ml} technique that uses labeled data to train a model.
The model learns from the training data and applies this learning to new, unseen data which is called test data.

% What is classification in machine learning
\section{Classification}

A classification model in machine learning is a type of predictive model that categorizes incoming input data into specific categories or classes.
It works by learning from a set of input features and corresponding labels during the training phase.
It then applies this learning to new, unseen data.
Classification models are used across various domains.
The output of a classification model is discrete, meaning it assigns each input to a specific category.
Examples of classification models include logistic regression, decision trees, random forests, and support vector machines.

\section{Univariate and Multivariate Time Series}
\begin{itemize}
    \item Differ in number of variables
    \item univariate: one observation recorded sequentially over time, e.g., temperature, stock prices; focus on understanding and forecasting a single variable's behavior
    \item multivariate: multiple observations recorded over the same time intervals, allowing for the analysis of interrelationships and interdependencies between these variables, e.g. temperature and humidity; focus on delving into understanding dynamic interactions and co-movements between multiple variables
\end{itemize}

\section{Temporal Dependency Handling}
Temporal dependency handling refers to the ability of a model to recognize and leverage the relationships or dependencies between data points that are separated by time. 
In time series data, the value at a given time point can be influenced by previous values, and understanding this dependency is crucial for accurate predictions.

%\noindent