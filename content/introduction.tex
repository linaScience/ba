\chapter{Introduction}\label{sec:intro}

% Darstellung des Themas der Arbeit
% genaue Auflistung der Fragestellungen (wieso Thema relevant?)
% knapper Überblick über Schritte der Problembehandlung:

% - Hinführung Thema
% - Herleitung und Ausformulierung der Fragestellung
% - Abgrenzung des Themas (Angabe von Aspekten, die zum Thema gehören, aber nicht behandelt/ausgeklammert werden)
% - Aufbau der Arbeit (Begründung der Gliederung)


In large-scale \ac{wifi} environments such as office buildings, shopping malls, and airports, where multiple \acp{ap} are required, people often move around indoors with their mobile devices.
To maintain a stable connection to the \ac{ssid} a device connects to, it must remain in the range of the \ac{ap} or may roam to another \ac{ap} with the same \ac{ssid}.
Roaming has been an essential feature of \ac{wifi} since the advent of the 802.11k\cite{802.11k} feature.
This process improved with \ac{ap}-initiated roaming, introduced in 802.11r\cite{802.11r}.
However, the current roaming process does not account for human movement patterns. 
For example, if a station is moving away from \ac{ap}1 towards \ac{ap}2, ideally, \ac{ap}2 should initiate the roaming process.
Furthermore, an AP may instruct a client device to roam based on signal strength without considering the device's trajectory or the user's likely destination.
The existing solutions need to be more robust and often lead to frequent hand-offs, inefficient resource usage, and in some cases, dropped connections.
This thesis will explore using time series \ac{ml} model to predict the top 3 \acp{bssid} a station may connect to next.
Therefore, this thesis needs data with \ac{wifi}, waypoint of clients and sensor data, e.g. acceleration.

A time series \ac{ml} models require time series data as input.
There are two possible data sources: generate new or utilize existing data. 
Data generation needs a comprehensive plan for accounting data setup and collection.
The generation is a time-consuming process and needs a lot of planning and evaluation beforehand.
Therefore, this thesis will utilize pre-existing data from a 2021 competition by Microsoft Research\cite{IndoorLocationNavigation}.
The data will be analyzed to determine what specific parts of the data we will use for the \ac{ml} model.

Furthermore, the data will be prepared for a time series \ac{ml} model.
After that, we will discuss the suitability of some pre-selected \ac{ml} models for the task. 
Due to the large number of data, this thesis will implement improving it, and evaluate the model's performance for a specific site and floor.

%\noindent