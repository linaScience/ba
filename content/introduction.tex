\chapter{Introduction}\label{sec:intro}

% Darstellung des Themas der Arbeit
% genaue Auflistung der Fragestellungen (wieso Thema relevant?)
% knapper Überblick über Schritte der Problembehandlung:

% - Hinführung Thema
% - Herleitung und Ausformulierung der Fragestellung
% - Abgrenzung des Themas (Angabe von Aspekten, die zum Thema gehören, aber nicht behandelt/ausgeklammert werden)
% - Aufbau der Arbeit (Begründung der Gliederung)

In big \ac{wifi} scenarios with multiple \acp{ap} in office buildings, shopping malls, and airports people are moving around indoors with their mobile phones and need to be connected to an \ac{ap}.
This is called roaming and is a very important part of Wi-Fi, since 802.11k\cite{802.11k}.
Roaming was improved with \ac{ap}-initiated roaming, which is a feature of 802.11r\cite{802.11r}.
But this roaming process does not include human movement, because if the \ac{ap}1 knows, that the station is passing it and moves towards \ac{ap}2, the \ac{ap}2 should initiate the roaming process.
This is not possible with the current roaming process, because the \ac{ap}1 does not know, that the station is moving towards \ac{ap}2.

Nowadays, using Machine learning to predict human movement in other scenarios like digital health or telecommunications is very common.
A prediction of the next \ac{ap} and initiate the roaming process before the station is moving towards the next \ac{ap}?
In this thesis, the prediction will be based on the \acp{rssi} of the \acp{bssid} and waypoints of the station.

We need to decide, if we want to generate the data on our own or use existing data.
We want to use machine learning to predict the next \ac{ap} with data generated by users devices.
Generating data on our own is not an option, because of the time and effort it would take to collect data.
So we decided to use existing data.
We will use a data set from kaggle\cite{kaggle} of a competition of Microsoft Research in 2021\cite{IndoorLocationNavigation}.
In this chapter, we will analyze the data.
Furthermore, we will identify which parts of the data is needed for the machine learning model.

There are several machine learning models which could be used.
We will take a look into some pre-selected models and discuss and propose a model, which could be the best for our data.
This thesis will not concept a new machine learning model and also will not use combined models.

The proposed decisions will be implemented.
As we need to preprocess the data for our model.
We will prefilter specific parts of the data, because as we found out in, not all data is needed for the machine learning model.
In this chapter, we will preprocess the data and implement the machine learning model as well as discuss the chosen encryption. 
A big part of machine learning models is tweaking the hyperparameters.

At the end we will evaluate the model and discuss the results.
We will also discuss, if the model could be used in a real-world scenario.

%\noindent
