\chapter{Introduction}\label{sec:intro}

% Darstellung des Themas der Arbeit
% genaue Auflistung der Fragestellungen (wieso Thema relevant?)
% knapper Überblick über Schritte der Problembehandlung:

% - Hinführung Thema
% - Herleitung und Ausformulierung der Fragestellung
% - Abgrenzung des Themas (Angabe von Aspekten, die zum Thema gehören, aber nicht behandelt/ausgeklammert werden)
% - Aufbau der Arbeit (Begründung der Gliederung)

% was Problem:
%    Roaming mostly initiated by AP with no knowledge of human movement. What if BSSIDs with the strongest signal could be predicted?
% wie Lösung: 
%   Machine Learning with data from a competition of microsoft in 2021 in China

In large-scale \ac{wifi} environments such as office buildings, shopping malls, and airports, where multiple \acp{ap} are required, people often move around indoors with their mobile devices.
To maintain a stable connection to the \ac{ssid} a device is connected to, it must remain in range of the \ac{ap} or may roam to another \ac{ap} with the same \ac{ssid}.
Roaming has been an essential feature of Wi-Fi since the advent of the 802.11k\cite{802.11k} feature.
This process was further enhanced with \ac{ap}-initiated roaming, introduced in 802.11r\cite{802.11r}.
However, the current roaming process doesn't account for human movement patterns. 
For example, if a station is moving away from \ac{ap}1 towards \ac{ap}2, ideally, \ac{ap}2 should initiate the roaming process.
Furthermore, an AP may instruct a client device to roam based on signal strength, without considering the device's trajectory or the user's likely destination.
The existing solutions are not robust enough and often leads to frequent hand-offs, inefficient resource usage, and in some cases, dropped connections.

- roam to specific APs, so BSSIDs with the strongest signal could be predicted
- 

% In this thesis, \ac{ml} will be used to predict the movement of a station in a \ac{wifi} scenario.
% In this thesis, the prediction will be based on the \acp{rssi} of the \acp{bssid} of the \ac{ap} and waypoints of the station.
\ac{ml} models, however, require data to function. 
There are two primary data sources: generate new data or utilize existing data. 
Data generation necessitates a comprehensive plan accounting for technologies, interferences, and data setup and collection—this can be a time-consuming process. 
Consequently, this study will utilize pre-existing data from a 2021 competition by Microsoft Research\cite{IndoorLocationNavigation}. 
The data will be analyzed to determine which segments are required for the ML model. 
Preprocessing is also necessary to convert raw sensor data into a format that the ML model can use.

%\noindent
