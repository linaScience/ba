\chapter{Introduction}\label{sec:intro}

% Darstellung des Themas der Arbeit
% genaue Auflistung der Fragestellungen (wieso Thema relevant?)
% knapper Überblick über Schritte der Problembehandlung:

% - Hinführung Thema
% - Herleitung und Ausformulierung der Fragestellung
% - Abgrenzung des Themas (Angabe von Aspekten, die zum Thema gehören, aber nicht behandelt/ausgeklammert werden)
% - Aufbau der Arbeit (Begründung der Gliederung)

% was Problem: Roaming mostly initiated by AP with no knowledge of human movement, 
% wie Lösung: Machine Learning 

In large-scale \ac{wifi} environments such as office buildings, shopping malls, and airports, where multiple \acp{ap} are required, people often move around indoors with their mobile devices.
To maintain a stable connection to the \ac{ssid} a device is connected to, it must remain in range of the \ac{ap} or may roam to another \ac{ap} with the same \ac{ssid}.
Roaming has been an essential feature of Wi-Fi since the advent of the 802.11k\cite{802.11k} feature.
This process was further enhanced with \ac{ap}-initiated roaming, introduced in 802.11r\cite{802.11r}.
However, the current roaming process doesn't account for human movement patterns. 
For example, if a station is moving away from \ac{ap}1 towards \ac{ap}2, ideally, \ac{ap}2 should initiate the roaming process.
Unfortunately, this is not feasible under the current scheme as \ac{ap}1 cannot detect the station'a movement to \ac{ap}2.

%\noindent
