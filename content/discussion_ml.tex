\chapter{Suitable Machine Learning Model}\label{ch:discuss-ml}

The floor I analyzed in \Cref{ch:data-ana} has 4795 \acp{bssid}.
Building upon the previous insights, the floor one Yintai City (Chengxi Branch) data has 4795 \acp{bssid}.
I will interpret each \ac{bssid} as a class, translating to a high-dimensional classification problem with 4795 classes.
This classification task necessitates a \ac{ml} model that can capture the underlying patterns in the data and generalize well to unseen instances.
I will discuss the classification models of \Cref{ch:background} by the following topics: Temporal Dependency Handling, Capacity and Complexity, Multivariate Data, Flexibility and Integration, and Regularization and Overfitting. 

\begin{description}
\item[Temporal Dependency Handling]
%Given the goal to predict the following \ac{bssid} based on the trajectory of a user, the \ac{ml} model needs to capture temporal dependencies in the data.
As mentioned in \Cref{sec:special-cases}, the irregularities in waypoint data further emphasize the need for a model that can handle such temporal structures.
While \acp{mlp} lack this capability \cite{mlp_and_nn}, \acp{hmm} do offer some temporal structure but often fall short for longer sequences and multivariate data \cite{hmm-rabiner-1989}.
Traditional \acp{rnn} are better equipped but have limitations, such as the vanishing gradient problem \cite{rnn_difficulties_2013}.
\acp{lstm}, in contrast, is specifically designed to handle long-term temporal dependencies, making them well-suited for our dataset with interpolated waypoints and \ac{wifi} data.
\end{description}

\begin{description}
\item[Capacity and Complexity]
Given the 4795 classes, the model needs to have a considerable capacity.
\acp{mlp} can scale their capacity by adding more hidden layers and units.
They are capable of modeling complex relationships within data through their nonlinear activations to differentiate between the subtle differences in patterns that might exist among them. 
\acp{hmm} have limitations in handling the complexity of multi-class and multivariate problems due to their inherent Markovian assumptions and discrete state representations. They may struggle with such a high-dimensional problem.
Traditional \acp{rnn} suffer from the vanishing gradient problem, especially in longer sequences, which limits their ability to capture long-term dependencies effectively \cite{rnn_difficulties_2013}.
\acp{lstm}, being deep learning models, can scale effectively in terms of capacity by adding more layers or units while still maintaining their ability to handle temporal data.
\end{description}

\begin{description}
\item [Multivariate Data]
The data at hand is multivariate, with features such as waypoint, accelerometer, and \ac{wifi} data.
While \acp{mlp} can handle multivariate data, they treat each feature and time step independently, often missing the interdependencies.
\acp{hmm} are primarily designed for univariate data. Extending them to multivariate scenarios requires additional complexities and assumptions.
\acp{rnn} and \acp{lstm} can seamlessly handle multivariate time series data. 
Their recurrent nature allows them to process each time step with multiple features effectively.
\end{description}

\begin{description}
\item[Flexibility and Integration]
Considering the interpolated data and the various preprocessing steps undertaken, as mentioned in \Cref{sec:prep-on-data-for-an-ml-model}, a flexible model that can be integrated with other architectures or preprocessing steps is desirable. 
\acp{mlp} are very flexible and can generally be used to learn a mapping from inputs to outputs \cite{mlp-vs-cnn-vs-rnn}.
This may be a good fit as I want the \ac{ml} to learn which BSSID is the next one.
\acp{hmm} are primarily designed for capturing state transitions in sequential data and may not be suitable for tasks requiring integrating spatial and temporal information.
Their rigid assumptions about state transitions limit their flexibility in capturing complex patterns \cite{hmm-rabiner-1989}.
Traditional \acp{rnn} can capture short-term dependencies and are relatively more uncomplicated to integrate with other architectures due to their sequential nature.
Also, \acp{lstm} can easily be integrated to capture temporal and spatial features. 
This flexibility of \acp{rnn} and \acp{lstm} is advantageous when dealing with complex and varied data sources.
\end{description}

\begin{description}
\item[Regularization and Overfitting]
The potential for overfitting is high given the large number of data points, as detailed in \Cref{tab:data_summary}, and the intricate relationships between them.
Therefore, dropouts may be used to prevent overfitting for each model \cite{srivastava14a}.
While \acp{rnn} might be more prone to overfitting \cite{rnn_difficulties_2013}, \acp{mlp}, \acp{hmm}, and \acp{lstm} offer better regularization capabilities.
\end{description}

\begin{table}[h]
    \centering
    \caption{Suitability of models for various requirements.}
    \begin{tabular}{l|c|c|c|c}
        & MLP & HMM & RNN & LSTM \\
        \hline
        Temporal Dependencies & & \checkmark & \checkmark & \checkmark \\
        Capacity & \checkmark & & & \checkmark \\
        Multivariate Data & & & \checkmark & \checkmark \\
        Flexibility & \checkmark & & \checkmark & \checkmark \\
        Overfitting & \checkmark & \checkmark & & \checkmark \\
    \end{tabular}
    \label{tab:model_suitability}
\end{table}


To summarize, while \acp{mlp}, \acp{hmm}, and traditional RNNs have their strengths and have been successful in many applications, they have problems with multivariate time series classification with many classes.
The classification problem for this thesis demands a model that can efficiently capture temporal patterns, scale in capacity, and handle multivariate data.
As \Cref{tab:model_suitability} shows, \acp{lstm} can deal with this challenge due to their unique architecture and properties, making them the preferred choice for this task and the selected model for our implementation.
