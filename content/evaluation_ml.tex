\chapter{Evaluation}\label{ch:evaluation}

For the evaluation of the model, I do the following:
First, the predictions of the test set are in \texttt{X\_test}.
Since the predictions are probabilities for each class, a conversion of the highest probability as true labels in \texttt{y\_test} is done, because this value will be the predicted \ac{ap}.
Those are one-hot encoded, as described in \Cref{ch:implementation}.
Then, I decode the one-hot encoded classes to the original classes, which can be done by \texttt{inverse\_transform} with the encoder.

Finally, I select the highest probabilities of the predictions\fmhkn{? ich dachte, da kommen nur 0/1 encodings raus? Wie können Sie dann probabilities aus der Prädiktion bekommen? ich bin verwirrt.} and compare it with the corresponding true label for the timestamp.
If they are equal, the prediction is correct, otherwise it is false.
The top three prediction accuracy of the \ac{lstm} model, so that one of the three predicts the correct \ac{bssid}, results\fmhkn{versteh ich nicht... die accuracy IST doch einfach 76 prozent? wie kann das was anderes sein, was dann in eine Zahl resultiert? verwirrend... } in 76\,\% %.
The predictions ranging from top one to top five are also evaluated, as shown in \Cref{fig:comparison_ml_heuristic_1_to_5}.

\begin{figure}[h]
    \centering
    \includegraphics*[scale=0.53]{images/comparison_ml_heuristic_1_to_5.pdf}
    \caption{Comparison of the Probabilities of correctly predicting the Top one to Top five \acp{ap} in the ML Model and the Heuristic Approach.}
    \label{fig:comparison_ml_heuristic_1_to_5}
\end{figure}

Instead of predicting the top three \acp{ap}, a heuristic approach could be used and will be compared with the model in the following.
The heuristic chooses the \ac{ap} with the highest signal strength as next \ac{ap}; if this is not the next strongest, it will proceed with the second strongest and so on.\fmhkn{versteh ich nicht... ? was beduet dieser Satz? }
\Cref{fig:comparison_ml_heuristic_1_to_5} shows the accuracy of this heuristic approach for choosing the strongest to the fifth strongest \ac{ap}.

Also, for predicting the top one to top five \ac{ap} the \ac{ml} models predictions are from top two on better than the heuristic approach.
The \ac{ml} model is slightly better than the heuristic approach for the top three \acp{ap}.\fmhkn{es tut mir leid, aber das ist mir zu kanpp. Ich kann hier nur mit Mühe folgen?}

\fmhkn{Sie müssen mir jetzt schon noch die Ergebnisse erklären. DAs reicht knapp, aber ein paar Sätze sollten es schon sein. Und spezifischer als der folgende Absatz.}

One of the major strengths of the machine learning model is that it utilizes user trajectories.
This approach can capture complex patterns and relationships that other models might overlook.
However, there are also inherent weaknesses.
The model has to deal with many classes (4795), and given that it only relies on six features, this may compromise its predictive accuracy.
This limitation might result in the model predicting worse than initially expected.