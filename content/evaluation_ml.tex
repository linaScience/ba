\chapter{Evaluation}\label{ch:evaluation}

For the evaluation of the model, I do the following:
First, the predictions of the test set are in \texttt{X\_test}.
Since the predictions are probabilities for each class, conversing the highest probability as true labels in \texttt{y\_test} is done because this value will be the predicted \ac{ap}.
Those are one-hot encoded, as described in \Cref{ch:implementation}.
Then, I decode the one-hot encoded classes to the original classes, which can be done by \texttt{inverse\_transform} with the encoder.

Finally, I select the target value of the predictions and compare it with the corresponding true label for the timestamp.
If they are equal, the prediction is correct, otherwise it is false.
The \threeAP prediction accuracy of the \ac{lstm} model, so that one of the three predicted \ac{bssid} is the one with the highest \ac{rssi} value, is 76\,\%.
The predictions ranging from a set size of one to five are also evaluated, as shown in \Cref{fig:comparison_ml_heuristic_1_to_5}.

\begin{figure}[h]
    \centering
    \includegraphics*[scale=0.53]{images/comparison_ml_heuristic_1_to_5.pdf}
    \caption{Comparison of the Probabilities of correctly predicting the set size of one to five \acp{ap} in the ML Model and the Heuristic Approach.}
    \label{fig:comparison_ml_heuristic_1_to_5}
\end{figure}

Instead of predicting \threeAP, a heuristic approach could be used and will be compared with the model in the following.
The heuristic chooses the latest \ac{ap} with the highest signal strength as following \ac{ap}.
If it is not the following highest \ac{rssi}, it will proceed with the latest second highest until the fifth highest to compare it to our approach. %\fmhkn{versteh ich nicht... ? was beduet dieser Satz? }
\Cref{fig:comparison_ml_heuristic_1_to_5} shows the accuracy of this heuristic approach for choosing the latest highest to the fifth highest \ac{ap}.

For predicting the set size of one, so the predicted \ac{ap} is correct, the heuristic's accuracy outperforms the \ac{ml} model by 3.5\,\%.
For the set size of two, the \ac{ml} model outperforms the heuristic by 0.3\,\%.
For \threeAP the prediction accuracy improves by 1.7\,\% compared to the heuristic approach.
Four and five set sizes outperform the heuristic by 2.2\,\%, and 1.2\,\%, respectively.
Training for malls with similar-sized floors would likely result in similar accuracies.

Regarding the execution time, the \ac{ml} model needs nearly 12 hours to tune on a M1 MacBook Pro.
The training of the proposed \ac{lstm} model takes about 25 minutes for a five-fold cross-validation.
The heuristic approach must only choose the three highest \ac{rssi} values out of 4795.
Choosing the three maximums of a set of 4795 values takes 75 milliseconds for one line of the prepared dataset, which is much faster than training the \ac{lstm} model.
Hence, the heuristic is much easier to implement and execute than the \ac{ml} model.

The \ac{ml} model needs to be trained for each mall floor separately, as the \ac{ap} are different on each floor.
The heuristic approach can be used for all floors, as it only needs the latest \ac{rssi} values of the \acp{ap}.

One of the significant strengths of the machine learning model is that it utilizes user trajectories.
However, the heuristic is a simpler and faster approach for predicting the following \ac{ap}.
This approach can capture complex patterns and relationships that other models might overlook.
However, there are also inherent weaknesses.
The model has to deal with 4795 classes, and given that it only relies on six features, this may compromise its predictive accuracy.
This limitation might result in the model predicting worse than initially expected.
